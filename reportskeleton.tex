\documentclass[a4]{article}
\usepackage{rsllisting}
\lstset{language=rsl}

%Comments in the Latex source are written after the % sign

\title{Course 02263 Mandatory Assignment 1, 2022} 

\author{name1 (study number1),\\ 
        name2 (study number2)}


\begin{document}

\maketitle

\tableofcontents
\newpage

\section{Introduction}
This document contains a solution to ... . 

\section{Requirements Specification}


\lstinputlisting{ColouringBasics.rsl}  

{\em Here you must informally explain the purpose of your auxiliary functions.}


\lstinputlisting{ColouringReq.rsl}  

{\em Here you must informally explain which requirements the pre
  condition expresses.}

{\em Here you must informally explain which requirements the post
  condition expresses.}

\section{Explicit Specification}

\lstinputlisting{ColouringEx.rsl}  

{\em Here you must  explain the idea behind your algorithm.}
{\em If you in this scheme introduce auxiliary functions in order to explicitly
  define the function, you must explain these as well.}

\section{Testing by Translation to SML}

\subsection{Test Specification}

\lstinputlisting{testColouringEx.rsl}
        
{\em Here you must  explain your test strategy and the purpose of the individual test
  cases.}


\subsection{Test Results}


The results of excecuting the SML translation of the RSL test cases are:

{\em Here you must  insert the output of the test run.}



\noindent From this we can conclude ...
{\em (tell whether the results are as expected)}.


\end{document}
